\documentclass[]{article}
\usepackage{lmodern}
\usepackage{amssymb,amsmath}
\usepackage{ifxetex,ifluatex}
\usepackage{fixltx2e} % provides \textsubscript
\ifnum 0\ifxetex 1\fi\ifluatex 1\fi=0 % if pdftex
  \usepackage[T1]{fontenc}
  \usepackage[utf8]{inputenc}
\else % if luatex or xelatex
  \ifxetex
    \usepackage{mathspec}
  \else
    \usepackage{fontspec}
  \fi
  \defaultfontfeatures{Ligatures=TeX,Scale=MatchLowercase}
\fi
% use upquote if available, for straight quotes in verbatim environments
\IfFileExists{upquote.sty}{\usepackage{upquote}}{}
% use microtype if available
\IfFileExists{microtype.sty}{%
\usepackage{microtype}
\UseMicrotypeSet[protrusion]{basicmath} % disable protrusion for tt fonts
}{}
\usepackage[margin=1in]{geometry}
\usepackage{hyperref}
\hypersetup{unicode=true,
            pdftitle={HMW 3- Data 621},
            pdfauthor={OMER OZEREN},
            pdfborder={0 0 0},
            breaklinks=true}
\urlstyle{same}  % don't use monospace font for urls
\usepackage{color}
\usepackage{fancyvrb}
\newcommand{\VerbBar}{|}
\newcommand{\VERB}{\Verb[commandchars=\\\{\}]}
\DefineVerbatimEnvironment{Highlighting}{Verbatim}{commandchars=\\\{\}}
% Add ',fontsize=\small' for more characters per line
\usepackage{framed}
\definecolor{shadecolor}{RGB}{248,248,248}
\newenvironment{Shaded}{\begin{snugshade}}{\end{snugshade}}
\newcommand{\AlertTok}[1]{\textcolor[rgb]{0.94,0.16,0.16}{#1}}
\newcommand{\AnnotationTok}[1]{\textcolor[rgb]{0.56,0.35,0.01}{\textbf{\textit{#1}}}}
\newcommand{\AttributeTok}[1]{\textcolor[rgb]{0.77,0.63,0.00}{#1}}
\newcommand{\BaseNTok}[1]{\textcolor[rgb]{0.00,0.00,0.81}{#1}}
\newcommand{\BuiltInTok}[1]{#1}
\newcommand{\CharTok}[1]{\textcolor[rgb]{0.31,0.60,0.02}{#1}}
\newcommand{\CommentTok}[1]{\textcolor[rgb]{0.56,0.35,0.01}{\textit{#1}}}
\newcommand{\CommentVarTok}[1]{\textcolor[rgb]{0.56,0.35,0.01}{\textbf{\textit{#1}}}}
\newcommand{\ConstantTok}[1]{\textcolor[rgb]{0.00,0.00,0.00}{#1}}
\newcommand{\ControlFlowTok}[1]{\textcolor[rgb]{0.13,0.29,0.53}{\textbf{#1}}}
\newcommand{\DataTypeTok}[1]{\textcolor[rgb]{0.13,0.29,0.53}{#1}}
\newcommand{\DecValTok}[1]{\textcolor[rgb]{0.00,0.00,0.81}{#1}}
\newcommand{\DocumentationTok}[1]{\textcolor[rgb]{0.56,0.35,0.01}{\textbf{\textit{#1}}}}
\newcommand{\ErrorTok}[1]{\textcolor[rgb]{0.64,0.00,0.00}{\textbf{#1}}}
\newcommand{\ExtensionTok}[1]{#1}
\newcommand{\FloatTok}[1]{\textcolor[rgb]{0.00,0.00,0.81}{#1}}
\newcommand{\FunctionTok}[1]{\textcolor[rgb]{0.00,0.00,0.00}{#1}}
\newcommand{\ImportTok}[1]{#1}
\newcommand{\InformationTok}[1]{\textcolor[rgb]{0.56,0.35,0.01}{\textbf{\textit{#1}}}}
\newcommand{\KeywordTok}[1]{\textcolor[rgb]{0.13,0.29,0.53}{\textbf{#1}}}
\newcommand{\NormalTok}[1]{#1}
\newcommand{\OperatorTok}[1]{\textcolor[rgb]{0.81,0.36,0.00}{\textbf{#1}}}
\newcommand{\OtherTok}[1]{\textcolor[rgb]{0.56,0.35,0.01}{#1}}
\newcommand{\PreprocessorTok}[1]{\textcolor[rgb]{0.56,0.35,0.01}{\textit{#1}}}
\newcommand{\RegionMarkerTok}[1]{#1}
\newcommand{\SpecialCharTok}[1]{\textcolor[rgb]{0.00,0.00,0.00}{#1}}
\newcommand{\SpecialStringTok}[1]{\textcolor[rgb]{0.31,0.60,0.02}{#1}}
\newcommand{\StringTok}[1]{\textcolor[rgb]{0.31,0.60,0.02}{#1}}
\newcommand{\VariableTok}[1]{\textcolor[rgb]{0.00,0.00,0.00}{#1}}
\newcommand{\VerbatimStringTok}[1]{\textcolor[rgb]{0.31,0.60,0.02}{#1}}
\newcommand{\WarningTok}[1]{\textcolor[rgb]{0.56,0.35,0.01}{\textbf{\textit{#1}}}}
\usepackage{graphicx,grffile}
\makeatletter
\def\maxwidth{\ifdim\Gin@nat@width>\linewidth\linewidth\else\Gin@nat@width\fi}
\def\maxheight{\ifdim\Gin@nat@height>\textheight\textheight\else\Gin@nat@height\fi}
\makeatother
% Scale images if necessary, so that they will not overflow the page
% margins by default, and it is still possible to overwrite the defaults
% using explicit options in \includegraphics[width, height, ...]{}
\setkeys{Gin}{width=\maxwidth,height=\maxheight,keepaspectratio}
\IfFileExists{parskip.sty}{%
\usepackage{parskip}
}{% else
\setlength{\parindent}{0pt}
\setlength{\parskip}{6pt plus 2pt minus 1pt}
}
\setlength{\emergencystretch}{3em}  % prevent overfull lines
\providecommand{\tightlist}{%
  \setlength{\itemsep}{0pt}\setlength{\parskip}{0pt}}
\setcounter{secnumdepth}{0}
% Redefines (sub)paragraphs to behave more like sections
\ifx\paragraph\undefined\else
\let\oldparagraph\paragraph
\renewcommand{\paragraph}[1]{\oldparagraph{#1}\mbox{}}
\fi
\ifx\subparagraph\undefined\else
\let\oldsubparagraph\subparagraph
\renewcommand{\subparagraph}[1]{\oldsubparagraph{#1}\mbox{}}
\fi

%%% Use protect on footnotes to avoid problems with footnotes in titles
\let\rmarkdownfootnote\footnote%
\def\footnote{\protect\rmarkdownfootnote}

%%% Change title format to be more compact
\usepackage{titling}

% Create subtitle command for use in maketitle
\newcommand{\subtitle}[1]{
  \posttitle{
    \begin{center}\large#1\end{center}
    }
}

\setlength{\droptitle}{-2em}

  \title{HMW 3- Data 621}
    \pretitle{\vspace{\droptitle}\centering\huge}
  \posttitle{\par}
    \author{OMER OZEREN}
    \preauthor{\centering\large\emph}
  \postauthor{\par}
    \date{}
    \predate{}\postdate{}
  
\usepackage{booktabs}
\usepackage{longtable}
\usepackage{array}
\usepackage{multirow}
\usepackage{wrapfig}
\usepackage{float}
\usepackage{colortbl}
\usepackage{pdflscape}
\usepackage{tabu}
\usepackage{threeparttable}
\usepackage{threeparttablex}
\usepackage[normalem]{ulem}
\usepackage{makecell}
\usepackage{xcolor}

\begin{document}
\maketitle

{
\setcounter{tocdepth}{5}
\tableofcontents
}
\hypertarget{summary}{%
\subsection{Summary}\label{summary}}

This report covers an attempt to build a binary logistic regression
model to predict whether the crime rate is above the median crime rate.
The model is based on a data set containing information on crime for
various Boston neighborhoods.

\hypertarget{data-exploration}{%
\subsection{Data Exploration}\label{data-exploration}}

The data set includes 466 observations with 12 variables (excluding the
target variable).

\hypertarget{summary-of-variables}{%
\paragraph{Summary of Variables}\label{summary-of-variables}}

\begin{verbatim}
##    Variable      Min    Median         Mean          SD      Max
## 1        zn   0.0000   0.00000  11.57725322  23.3646511 100.0000
## 2     indus   0.4600   9.69000  11.10502146   6.8458549  27.7400
## 3      chas   0.0000   0.00000   0.07081545   0.2567920   1.0000
## 4       nox   0.3890   0.53800   0.55431052   0.1166667   0.8710
## 5        rm   3.8630   6.21000   6.29067382   0.7048513   8.7800
## 6       age   2.9000  77.15000  68.36759657  28.3213784 100.0000
## 7       dis   1.1296   3.19095   3.79569292   2.1069496  12.1265
## 8       rad   1.0000   5.00000   9.53004292   8.6859272  24.0000
## 9       tax 187.0000 334.50000 409.50214592 167.9000887 711.0000
## 10  ptratio  12.6000  18.90000  18.39849785   2.1968447  22.0000
## 11    lstat   1.7300  11.35000  12.63145923   7.1018907  37.9700
## 12     medv   5.0000  21.20000  22.58927039   9.2396814  50.0000
## 13   target   0.0000   0.00000   0.49141631   0.5004636   1.0000
##    Num of NAs Num of Zeros
## 1           0          339
## 2           0            0
## 3           0          433
## 4           0            0
## 5           0            0
## 6           0            0
## 7           0            0
## 8           0            0
## 9           0            0
## 10          0            0
## 11          0            0
## 12          0            0
## 13          0          237
\end{verbatim}

\hypertarget{independent-variables}{%
\paragraph{Independent Variables}\label{independent-variables}}

\begin{itemize}
\tightlist
\item
  \texttt{zn} - \emph{proportion of residential land zoned for large
  lots (over 25,000 square feet)} - 339 out of 466 (or about 76\%) of
  observations have a value of 0. It is possible that majority of
  neighborhoods will not have any residential land zoned for large lots.
  Therefore, it is likely that 0 represents a valid value rather than a
  missing one.
\item
  \texttt{indus} - \emph{proportion of non-retail business acres per
  suburb}
\item
  \texttt{chas} - \emph{a dummy variable for whether the suburb borders
  the Charles River (1) or not (0)} - 433 out of 466 (or about 92.9\%)
  of observations have a value of 0. Even though the Charles River is a
  promimnent feature of the Boston area, it is quite reasonable to
  assume that most neighborhoods do not border the river.
\item
  \texttt{nox} - \emph{nitrogen oxides concentration (parts per 10
  million)} - Looking at the scatterplot there seems to be some
  correlation between the nitrogen oxides concentration and the target
  variable.
\end{itemize}

\includegraphics{OMEROZEREN_HMW_3_files/figure-latex/unnamed-chunk-2-1.pdf}

\begin{itemize}
\tightlist
\item
  \texttt{rm} - \emph{average number of rooms per dwelling} - Because
  this is an average of number of rooms, this is a continous variable.
\item
  \texttt{age} - \emph{proportion of owner-occupied units built prior to
  1940} - There is nothing unusual about this variable; however, it is
  interesting to note that the mean of 68.37 and median of 77.15 shows
  that there is a large number of pre-war buildings. Not surprising for
  an old city like Boston.
\item
  \texttt{dis} - \emph{weighted mean of distances to five Boston
  employment centers} - Majority of observations above the median crime
  rate are within 5 miles of an employment center (there are only 2
  observations over 5 miles away). And there may be some correlation
  between distance and the target variable.
\item
  \texttt{rad} - \emph{index of accessibility to radial highways} - This
  is a discrete variable with 9 different values in the observations (1
  through 8 and 24). The smallest bucket is \texttt{rad} value of 7 with
  15 observations.
\item
  \texttt{tax} - \emph{full-value property-tax rate per \$10,000}
\item
  \texttt{ptratio} - \emph{pupil-teacher ratio by town}
\item
  \texttt{lstat} - \emph{lower status of the population (percent)}
\item
  \texttt{medv} - \emph{median value of owner-occupied homes in \$1000s}
\end{itemize}

\hypertarget{targetdependent-variable}{%
\paragraph{Target/Dependent Variable}\label{targetdependent-variable}}

\begin{itemize}
\tightlist
\item
  \texttt{target} - \emph{whether the crime rate is above the median
  crime rate (1) or not (0)} - There are 237 observation with
  \texttt{target} value of 0 and 229 observations with \texttt{target}
  value of 1 making it about 50/50 split, or more precisely there are
  \textbf{50.86\% of 0s and 49.14\% of 1s}.
\end{itemize}

\hypertarget{correlation-matrix}{%
\paragraph{Correlation Matrix}\label{correlation-matrix}}

Below is the correlation matrix for the data set.

\begin{itemize}
\tightlist
\item
  There is a very high correlation (0.91) between \texttt{tax} and
  \texttt{rad}. Meaning behind \texttt{rad} values/categories is not
  explicitly specified. It may be that the higher the highway
  accessibility is, the higher property taxes are. Alternatively, radial
  highways and higher property taxes may signify suburbs while lack of
  radial highways may imply inner city (with possibly poorer, lower
  taxed properties).
\item
  \texttt{nox} has the highest correlation with the target variable, but
  \texttt{age}, \texttt{dis}, \texttt{rad} and \texttt{tax} are also
  fairly highly correlated to \texttt{target} (above 0.6).
\item
  The following pairs have correlation at or above 0.7 (or below -0.7 in
  case of negative correlation): \texttt{nox}/\texttt{indus},
  \texttt{dis}/\texttt{indus}, \texttt{tax}/\texttt{indus},
  \texttt{age}/\texttt{nox}, \texttt{dis}/\texttt{nox},
  \texttt{medv}/\texttt{rm}, \texttt{dis}/\texttt{age} and
  \texttt{medv}/\texttt{lstat}.
\end{itemize}

\begin{table}[H]
\centering
\begin{tabular}{l|c|c|c|c|c|c|c|c|c|c|c|c|c}
\hline
  & zn & indus & chas & nox & rm & age & dis & rad & tax & ptratio & lstat & medv & target\\
\hline
zn & \textcolor{blue}{1} & \textcolor{black}{-0.54} & \textcolor{black}{-0.04} & \textcolor{black}{-0.52} & \textcolor{black}{0.32} & \textcolor{black}{-0.57} & \textcolor{black}{0.66} & \textcolor{black}{-0.32} & \textcolor{black}{-0.32} & \textcolor{black}{-0.39} & \textcolor{black}{-0.43} & \textcolor{black}{0.38} & \textcolor{black}{-0.43}\\
\hline
indus & \textcolor{black}{-0.54} & \textcolor{blue}{1} & \textcolor{black}{0.06} & \textcolor{blue}{0.76} & \textcolor{black}{-0.39} & \textcolor{black}{0.64} & \textcolor{blue}{-0.7} & \textcolor{black}{0.6} & \textcolor{blue}{0.73} & \textcolor{black}{0.39} & \textcolor{black}{0.61} & \textcolor{black}{-0.5} & \textcolor{black}{0.6}\\
\hline
chas & \textcolor{black}{-0.04} & \textcolor{black}{0.06} & \textcolor{blue}{1} & \textcolor{black}{0.1} & \textcolor{black}{0.09} & \textcolor{black}{0.08} & \textcolor{black}{-0.1} & \textcolor{black}{-0.02} & \textcolor{black}{-0.05} & \textcolor{black}{-0.13} & \textcolor{black}{-0.05} & \textcolor{black}{0.16} & \textcolor{black}{0.08}\\
\hline
nox & \textcolor{black}{-0.52} & \textcolor{blue}{0.76} & \textcolor{black}{0.1} & \textcolor{blue}{1} & \textcolor{black}{-0.3} & \textcolor{blue}{0.74} & \textcolor{blue}{-0.77} & \textcolor{black}{0.6} & \textcolor{black}{0.65} & \textcolor{black}{0.18} & \textcolor{black}{0.6} & \textcolor{black}{-0.43} & \textcolor{blue}{0.73}\\
\hline
rm & \textcolor{black}{0.32} & \textcolor{black}{-0.39} & \textcolor{black}{0.09} & \textcolor{black}{-0.3} & \textcolor{blue}{1} & \textcolor{black}{-0.23} & \textcolor{black}{0.2} & \textcolor{black}{-0.21} & \textcolor{black}{-0.3} & \textcolor{black}{-0.36} & \textcolor{black}{-0.63} & \textcolor{blue}{0.71} & \textcolor{black}{-0.15}\\
\hline
age & \textcolor{black}{-0.57} & \textcolor{black}{0.64} & \textcolor{black}{0.08} & \textcolor{blue}{0.74} & \textcolor{black}{-0.23} & \textcolor{blue}{1} & \textcolor{blue}{-0.75} & \textcolor{black}{0.46} & \textcolor{black}{0.51} & \textcolor{black}{0.26} & \textcolor{black}{0.61} & \textcolor{black}{-0.38} & \textcolor{black}{0.63}\\
\hline
dis & \textcolor{black}{0.66} & \textcolor{blue}{-0.7} & \textcolor{black}{-0.1} & \textcolor{blue}{-0.77} & \textcolor{black}{0.2} & \textcolor{blue}{-0.75} & \textcolor{blue}{1} & \textcolor{black}{-0.49} & \textcolor{black}{-0.53} & \textcolor{black}{-0.23} & \textcolor{black}{-0.51} & \textcolor{black}{0.26} & \textcolor{black}{-0.62}\\
\hline
rad & \textcolor{black}{-0.32} & \textcolor{black}{0.6} & \textcolor{black}{-0.02} & \textcolor{black}{0.6} & \textcolor{black}{-0.21} & \textcolor{black}{0.46} & \textcolor{black}{-0.49} & \textcolor{blue}{1} & \textcolor{blue}{0.91} & \textcolor{black}{0.47} & \textcolor{black}{0.5} & \textcolor{black}{-0.4} & \textcolor{black}{0.63}\\
\hline
tax & \textcolor{black}{-0.32} & \textcolor{blue}{0.73} & \textcolor{black}{-0.05} & \textcolor{black}{0.65} & \textcolor{black}{-0.3} & \textcolor{black}{0.51} & \textcolor{black}{-0.53} & \textcolor{blue}{0.91} & \textcolor{blue}{1} & \textcolor{black}{0.47} & \textcolor{black}{0.56} & \textcolor{black}{-0.49} & \textcolor{black}{0.61}\\
\hline
ptratio & \textcolor{black}{-0.39} & \textcolor{black}{0.39} & \textcolor{black}{-0.13} & \textcolor{black}{0.18} & \textcolor{black}{-0.36} & \textcolor{black}{0.26} & \textcolor{black}{-0.23} & \textcolor{black}{0.47} & \textcolor{black}{0.47} & \textcolor{blue}{1} & \textcolor{black}{0.38} & \textcolor{black}{-0.52} & \textcolor{black}{0.25}\\
\hline
lstat & \textcolor{black}{-0.43} & \textcolor{black}{0.61} & \textcolor{black}{-0.05} & \textcolor{black}{0.6} & \textcolor{black}{-0.63} & \textcolor{black}{0.61} & \textcolor{black}{-0.51} & \textcolor{black}{0.5} & \textcolor{black}{0.56} & \textcolor{black}{0.38} & \textcolor{blue}{1} & \textcolor{blue}{-0.74} & \textcolor{black}{0.47}\\
\hline
medv & \textcolor{black}{0.38} & \textcolor{black}{-0.5} & \textcolor{black}{0.16} & \textcolor{black}{-0.43} & \textcolor{blue}{0.71} & \textcolor{black}{-0.38} & \textcolor{black}{0.26} & \textcolor{black}{-0.4} & \textcolor{black}{-0.49} & \textcolor{black}{-0.52} & \textcolor{blue}{-0.74} & \textcolor{blue}{1} & \textcolor{black}{-0.27}\\
\hline
target & \textcolor{black}{-0.43} & \textcolor{black}{0.6} & \textcolor{black}{0.08} & \textcolor{blue}{0.73} & \textcolor{black}{-0.15} & \textcolor{black}{0.63} & \textcolor{black}{-0.62} & \textcolor{black}{0.63} & \textcolor{black}{0.61} & \textcolor{black}{0.25} & \textcolor{black}{0.47} & \textcolor{black}{-0.27} & \textcolor{blue}{1}\\
\hline
\end{tabular}
\end{table}

\hypertarget{scatterplot-matrix}{%
\paragraph{Scatterplot Matrix}\label{scatterplot-matrix}}

\includegraphics{OMEROZEREN_HMW_3_files/figure-latex/unnamed-chunk-4-1.pdf}

\begin{itemize}
\tightlist
\item
  Reviewing the scatterplot matrix shows several pairs with possible
  relationships. Two most prominent are \texttt{nox}/\texttt{dis} and
  \texttt{lstat}/\texttt{medv}.
\item
  \texttt{rm} and \texttt{medv} may have a linear relationship as well.
\item
  \texttt{rad} and \texttt{tax} have a very prominent outlier. Further
  inspection of data shows that all observations with \texttt{rad} value
  of 24 have a \texttt{tax} value of 666, so the outlier is actually
  multiple observations mapped to the same spot. Interestingly, all of
  these obervations also have a \texttt{target} value of 1. This
  combination may warrant closer inspection.
\end{itemize}

Above data analysis treats \texttt{rad} and \texttt{chas} as numeric
variables; however, treating them as categorical variables may better
reflect the nature of those variables, so for modelling they will be
converted to factors.

\hypertarget{modelling}{%
\subsection{Modelling}\label{modelling}}

The dependent variable, \texttt{target}, is binary. For this project it
is assumed that observations are independent of each other as there is
no reason to believe otherwise.

As the first step, in order to test and compare performance of various
models, data was split into training (75\%) and testing (25\%) sets. The
training set includes 350 randomly chosen observations and the testing
set includes 116 remaining observations.

\hypertarget{model-1-variables-with-high-correlation-to-target-variable}{%
\paragraph{Model 1: Variables with High Correlation to Target
Variable}\label{model-1-variables-with-high-correlation-to-target-variable}}

The first model includes 5 variables with the highest correlation
coefficients when compared agains the target variable. This simple model
will allow for testing methodology as well as corresponding R code.

\begin{verbatim}
## 
## Call:
## glm(formula = target ~ nox + age + dis + rad + tax, family = binomial(link = "logit"), 
##     data = crimeTRAIN)
## 
## Deviance Residuals: 
##     Min       1Q   Median       3Q      Max  
## -3.0293  -0.1272   0.0000   0.0001   3.2135  
## 
## Coefficients:
##               Estimate Std. Error z value Pr(>|z|)    
## (Intercept) -4.870e+01  3.499e+03  -0.014    0.989    
## nox          6.026e+01  1.220e+01   4.939 7.84e-07 ***
## age          2.782e-03  1.372e-02   0.203    0.839    
## dis          7.224e-03  2.592e-01   0.028    0.978    
## rad2        -2.036e+00  4.801e+03   0.000    1.000    
## rad3         2.044e+01  3.499e+03   0.006    0.995    
## rad4         2.262e+01  3.499e+03   0.006    0.995    
## rad5         2.018e+01  3.499e+03   0.006    0.995    
## rad6         1.846e+01  3.499e+03   0.005    0.996    
## rad7         2.294e+01  3.499e+03   0.007    0.995    
## rad8         2.544e+01  3.499e+03   0.007    0.994    
## rad24        4.349e+01  3.773e+03   0.012    0.991    
## tax         -1.656e-02  3.910e-03  -4.235 2.28e-05 ***
## ---
## Signif. codes:  0 '***' 0.001 '**' 0.01 '*' 0.05 '.' 0.1 ' ' 1
## 
## (Dispersion parameter for binomial family taken to be 1)
## 
##     Null deviance: 485.100  on 349  degrees of freedom
## Residual deviance:  97.662  on 337  degrees of freedom
## AIC: 123.66
## 
## Number of Fisher Scoring iterations: 19
\end{verbatim}

\begin{verbatim}
##           Reference
## Prediction  0  1
##          0 56  3
##          1  4 53
\end{verbatim}

\begin{verbatim}
## fitting null model for pseudo-r2
\end{verbatim}

Model summary and confusion matrix of running this model against test
data are above. The accuracy rate (0.9396552) is very good and the
McFadden R\^{}2 value (0.7986761) is also high. AIC value is 123.66.
Additionally, consider the ROC curve for this model.

\includegraphics{OMEROZEREN_HMW_3_files/figure-latex/unnamed-chunk-8-1.pdf}

Area under the curve is 0.9815641.

\hypertarget{model-2-all-variables}{%
\paragraph{Model 2: All Variables}\label{model-2-all-variables}}

The second model includes all 12 available independent variables.

\begin{verbatim}
## 
## Call:
## glm(formula = target ~ ., family = binomial(link = "logit"), 
##     data = crimeTRAIN)
## 
## Deviance Residuals: 
##     Min       1Q   Median       3Q      Max  
## -2.6329  -0.0803   0.0000   0.0001   4.1121  
## 
## Coefficients:
##               Estimate Std. Error z value Pr(>|z|)    
## (Intercept) -5.294e+01  3.400e+03  -0.016   0.9876    
## zn          -1.444e-01  7.044e-02  -2.051   0.0403 *  
## indus       -1.363e-01  1.221e-01  -1.116   0.2644    
## chas1       -1.543e+00  1.480e+00  -1.042   0.2973    
## nox          6.142e+01  1.452e+01   4.230 2.34e-05 ***
## rm          -1.867e-01  1.262e+00  -0.148   0.8824    
## age          1.133e-02  1.837e-02   0.617   0.5372    
## dis          3.650e-01  2.982e-01   1.224   0.2210    
## rad2        -4.535e-01  4.821e+03   0.000   0.9999    
## rad3         1.738e+01  3.400e+03   0.005   0.9959    
## rad4         2.143e+01  3.400e+03   0.006   0.9950    
## rad5         1.873e+01  3.400e+03   0.006   0.9956    
## rad6         1.647e+01  3.400e+03   0.005   0.9961    
## rad7         2.451e+01  3.400e+03   0.007   0.9942    
## rad8         2.464e+01  3.400e+03   0.007   0.9942    
## rad24        4.091e+01  3.695e+03   0.011   0.9912    
## tax         -9.692e-03  6.242e-03  -1.553   0.1205    
## ptratio     -1.060e-02  2.141e-01  -0.050   0.9605    
## lstat        7.859e-02  7.283e-02   1.079   0.2806    
## medv         1.320e-01  1.157e-01   1.141   0.2540    
## ---
## Signif. codes:  0 '***' 0.001 '**' 0.01 '*' 0.05 '.' 0.1 ' ' 1
## 
## (Dispersion parameter for binomial family taken to be 1)
## 
##     Null deviance: 485.100  on 349  degrees of freedom
## Residual deviance:  88.097  on 330  degrees of freedom
## AIC: 128.1
## 
## Number of Fisher Scoring iterations: 19
\end{verbatim}

\begin{verbatim}
##           Reference
## Prediction  0  1
##          0 58  1
##          1  5 52
\end{verbatim}

\begin{verbatim}
## fitting null model for pseudo-r2
\end{verbatim}

Model summary and confusion matrix of running this model against test
data are above. The accuracy rate (0.9482759) is very good and the
McFadden R\^{}2 value (0.8183936) is also high. AIC value is 128.1.
Additionally, consider the ROC curve for this model.

\includegraphics{OMEROZEREN_HMW_3_files/figure-latex/unnamed-chunk-10-1.pdf}

Area under the curve is 0.9854297.

Comparing to the first model AIC has slightly increased (worse), but
accuracy, McFadden R\^{}2 and AUC all also slightly increased (better).

\hypertarget{model-3-stepaic-method}{%
\paragraph{\texorpdfstring{Model 3: \emph{StepAIC}
Method}{Model 3: StepAIC Method}}\label{model-3-stepaic-method}}

The third model starts with all 12 available independent variables, but
then drops them one by one using the stepwise algorithm.

\begin{verbatim}
## 
## Call:
## glm(formula = target ~ zn + nox + rad + tax, family = binomial(link = "logit"), 
##     data = crimeTRAIN)
## 
## Deviance Residuals: 
##      Min        1Q    Median        3Q       Max  
## -2.94915  -0.13567   0.00000   0.00011   3.15933  
## 
## Coefficients:
##               Estimate Std. Error z value Pr(>|z|)    
## (Intercept) -4.648e+01  3.400e+03  -0.014    0.989    
## zn          -6.837e-02  4.509e-02  -1.516    0.129    
## nox          5.669e+01  9.605e+00   5.902 3.59e-09 ***
## rad2        -2.026e+00  4.728e+03   0.000    1.000    
## rad3         2.013e+01  3.400e+03   0.006    0.995    
## rad4         2.244e+01  3.400e+03   0.007    0.995    
## rad5         2.015e+01  3.400e+03   0.006    0.995    
## rad6         1.830e+01  3.400e+03   0.005    0.996    
## rad7         2.465e+01  3.400e+03   0.007    0.994    
## rad8         2.573e+01  3.400e+03   0.008    0.994    
## rad24        4.309e+01  3.684e+03   0.012    0.991    
## tax         -1.602e-02  3.747e-03  -4.277 1.90e-05 ***
## ---
## Signif. codes:  0 '***' 0.001 '**' 0.01 '*' 0.05 '.' 0.1 ' ' 1
## 
## (Dispersion parameter for binomial family taken to be 1)
## 
##     Null deviance: 485.100  on 349  degrees of freedom
## Residual deviance:  94.991  on 338  degrees of freedom
## AIC: 118.99
## 
## Number of Fisher Scoring iterations: 19
\end{verbatim}

\begin{verbatim}
##           Reference
## Prediction  0  1
##          0 56  3
##          1  4 53
\end{verbatim}

\begin{verbatim}
## fitting null model for pseudo-r2
\end{verbatim}

Model summary and confusion matrix of running this model against test
data are above. The accuracy rate (0.9396552) is very good and the
McFadden R\^{}2 value (0.8041826) is also high. AIC value is 118.99.
Additionally, consider the ROC curve for this model.

\includegraphics{OMEROZEREN_HMW_3_files/figure-latex/unnamed-chunk-12-1.pdf}

Area under the curve is 0.9849836.

The third model has the best (lowest) AIC value (better). Accuracy is
the same as for the first model (and slightly lower than the second
model). AUC is lower, but very close to the AUC value for the second
model. Finally, McFadden R\^{}2 falls between the first and second
models, but the change is also very small.

\hypertarget{additional-models}{%
\paragraph{Additional Models}\label{additional-models}}

Basic models produced very efficient results. Several other models were
attempted, but they did not produce significant improvements and
therefore simplier, easier to interpret basic models were preferred.
Other models included variable transformations and variable
interactions. Since this project does not deals with critical and
sensitive data with high cost of errors, such as medical or national
security projects may, the accuracy of the basic models is deemed
acceptable.

\hypertarget{model-selection}{%
\subsection{Model Selection}\label{model-selection}}

All 3 models generated good overall results, but the third model
(\emph{StepAIC} model) is chosen for its simplicity. It is important to
note that even though general parameters between models are close, one
may be preferred over the other based on application. For example, the
second and third models have similar number of errors (6 and 7);
however, the second model has more Type II/False Negative errors and
less Type I/False Positive errors than the third model. This difference
in sensitivity and specificity may be important for some applications.

Additionally, one small adjustment to the model is to convert
\texttt{nox} from parts per 10 million to parts per 100,000. This will
help interpret the model coefficients.

\begin{verbatim}
## 
## Call:
## glm(formula = target ~ zn + I(nox * 100) + rad + tax, family = binomial(link = "logit"), 
##     data = crime)
## 
## Deviance Residuals: 
##     Min       1Q   Median       3Q      Max  
## -2.9406  -0.1155   0.0000   0.0001   3.3805  
## 
## Coefficients:
##                Estimate Std. Error z value Pr(>|z|)    
## (Intercept)  -4.640e+01  3.102e+03  -0.015   0.9881    
## zn           -8.837e-02  4.308e-02  -2.051   0.0402 *  
## I(nox * 100)  5.643e-01  8.038e-02   7.021 2.21e-12 ***
## rad2         -1.885e+00  4.225e+03   0.000   0.9996    
## rad3          1.987e+01  3.102e+03   0.006   0.9949    
## rad4          2.255e+01  3.102e+03   0.007   0.9942    
## rad5          2.014e+01  3.102e+03   0.006   0.9948    
## rad6          1.865e+01  3.102e+03   0.006   0.9952    
## rad7          2.631e+01  3.102e+03   0.008   0.9932    
## rad8          2.599e+01  3.102e+03   0.008   0.9933    
## rad24         4.410e+01  3.322e+03   0.013   0.9894    
## tax          -1.614e-02  3.094e-03  -5.218 1.81e-07 ***
## ---
## Signif. codes:  0 '***' 0.001 '**' 0.01 '*' 0.05 '.' 0.1 ' ' 1
## 
## (Dispersion parameter for binomial family taken to be 1)
## 
##     Null deviance: 645.88  on 465  degrees of freedom
## Residual deviance: 128.29  on 454  degrees of freedom
## AIC: 152.29
## 
## Number of Fisher Scoring iterations: 19
\end{verbatim}

\hypertarget{model-performance-and-description}{%
\subsection{Model Performance and
Description}\label{model-performance-and-description}}

\hypertarget{k-fold-cross-validation}{%
\paragraph{K-Fold Cross Validation}\label{k-fold-cross-validation}}

To assess the performance of selected model, below are results of
10-fold cross-validation. The model performs as expected.

\begin{verbatim}
## Confusion Matrix and Statistics
## 
##           Reference
## Prediction  0  1
##          0 56  3
##          1  4 53
##                                           
##                Accuracy : 0.9397          
##                  95% CI : (0.8796, 0.9754)
##     No Information Rate : 0.5172          
##     P-Value [Acc > NIR] : <2e-16          
##                                           
##                   Kappa : 0.8792          
##                                           
##  Mcnemar's Test P-Value : 1               
##                                           
##             Sensitivity : 0.9333          
##             Specificity : 0.9464          
##          Pos Pred Value : 0.9492          
##          Neg Pred Value : 0.9298          
##              Prevalence : 0.5172          
##          Detection Rate : 0.4828          
##    Detection Prevalence : 0.5086          
##       Balanced Accuracy : 0.9399          
##                                           
##        'Positive' Class : 0               
## 
\end{verbatim}

\hypertarget{deviance}{%
\paragraph{Deviance}\label{deviance}}

Similarly, the deviance table below demonstrated that each variable
significantly contributes to the drop in deviance difference.

\begin{verbatim}
## Analysis of Deviance Table
## 
## Model: binomial, link: logit
## 
## Response: target
## 
## Terms added sequentially (first to last)
## 
## 
##              Df Deviance Resid. Df Resid. Dev  Pr(>Chi)    
## NULL                           465     645.88              
## zn            1  127.411       464     518.46 < 2.2e-16 ***
## I(nox * 100)  1  230.177       463     288.29 < 2.2e-16 ***
## rad           8  127.537       455     160.75 < 2.2e-16 ***
## tax           1   32.462       454     128.29 1.216e-08 ***
## ---
## Signif. codes:  0 '***' 0.001 '**' 0.01 '*' 0.05 '.' 0.1 ' ' 1
\end{verbatim}

\hypertarget{variance-inflation-factor}{%
\paragraph{Variance Inflation Factor}\label{variance-inflation-factor}}

\begin{verbatim}
##                  GVIF Df GVIF^(1/(2*Df))
## zn           2.866869  1        1.693183
## I(nox * 100) 2.953331  1        1.718526
## rad          5.114539  8        1.107390
## tax          2.154850  1        1.467941
\end{verbatim}

VIFs are reasonable, so that we can assume that there is not much
multicollinearity between variables.

\hypertarget{histogram-of-predicted-probabilities}{%
\paragraph{Histogram of Predicted
Probabilities}\label{histogram-of-predicted-probabilities}}

Distribution of predicted probabilities generated by running all
training data through the model shows that the model is predicting 0 or
1 with high probability. There are few instances where probability shows
less certainty in the selected outcome.

\includegraphics{OMEROZEREN_HMW_3_files/figure-latex/unnamed-chunk-17-1.pdf}

\hypertarget{coefficientsoddsvariable-contribution}{%
\paragraph{Coefficients/Odds/Variable
Contribution}\label{coefficientsoddsvariable-contribution}}

\begin{verbatim}
##              exp(model$coefficients)
## (Intercept)             7.069802e-21
## zn                      9.154199e-01
## I(nox * 100)            1.758185e+00
## rad2                    1.517949e-01
## rad3                    4.269957e+08
## rad4                    6.223348e+09
## rad5                    5.589707e+08
## rad6                    1.256121e+08
## rad7                    2.665850e+11
## rad8                    1.932043e+11
## rad24                   1.419253e+19
## tax                     9.839877e-01
\end{verbatim}

For \texttt{zn}, the coefficient is negative and the odds of having an
above median crime rate is 0.9154. Higher \texttt{zn}, meaning more
large lots, is less likely to increase the crime rate.

For \texttt{nox}, the coefficient is positive and the odds is 1.75, so
there is a 75\% increase in odds with higher \texttt{nox} values. Higher
levels of nitrogen oxide indicate more congested neighborhoods. It is
possible to theorize that more urban, congested areas are more likely to
have higher crime than suburban areas.

For \texttt{tax}, the coefficient is negative and the odds is 0.984.
Decrease in crime rate is more likely with the increase of property-tax
rates.

Finally, for \texttt{rad} all coefficients are positive except for
\texttt{rad} value of 2. There is no explicit explanation for values of
\texttt{rad} variable. Assuming that low values mean higher
accessibility to radial highways, it is possible to theorize that living
close, but not too close to highways is more likely to decrease the
crime rate, but then moving away from highways is more likely to
increase it. Odds are difficult to intepret possibly because of outliers
(most likely \texttt{rad} value of 24).

\hypertarget{evaluation-data-set}{%
\subsection{Evaluation Data Set}\label{evaluation-data-set}}

\begin{verbatim}
##    zn indus chas   nox    rm   age    dis rad tax ptratio lstat medv
## 1   0  7.07    0 0.469 7.185  61.1 4.9671   2 242    17.8  4.03 34.7
## 2   0  8.14    0 0.538 6.096  84.5 4.4619   4 307    21.0 10.26 18.2
## 3   0  8.14    0 0.538 6.495  94.4 4.4547   4 307    21.0 12.80 18.4
## 4   0  8.14    0 0.538 5.950  82.0 3.9900   4 307    21.0 27.71 13.2
## 5   0  5.96    0 0.499 5.850  41.5 3.9342   5 279    19.2  8.77 21.0
## 6  25  5.13    0 0.453 5.741  66.2 7.2254   8 284    19.7 13.15 18.7
## 7  25  5.13    0 0.453 5.966  93.4 6.8185   8 284    19.7 14.44 16.0
## 8   0  4.49    0 0.449 6.630  56.1 4.4377   3 247    18.5  6.53 26.6
## 9   0  4.49    0 0.449 6.121  56.8 3.7476   3 247    18.5  8.44 22.2
## 10  0  2.89    0 0.445 6.163  69.6 3.4952   2 276    18.0 11.34 21.4
## 11  0 25.65    0 0.581 5.856  97.0 1.9444   2 188    19.1 25.41 17.3
## 12  0 25.65    0 0.581 5.613  95.6 1.7572   2 188    19.1 27.26 15.7
## 13  0 21.89    0 0.624 5.637  94.7 1.9799   4 437    21.2 18.34 14.3
## 14  0 19.58    0 0.605 6.101  93.0 2.2834   5 403    14.7  9.81 25.0
## 15  0 19.58    0 0.605 5.880  97.3 2.3887   5 403    14.7 12.03 19.1
## 16  0 10.59    1 0.489 5.960  92.1 3.8771   4 277    18.6 17.27 21.7
## 17  0  6.20    0 0.504 6.552  21.4 3.3751   8 307    17.4  3.76 31.5
## 18  0  6.20    0 0.507 8.247  70.4 3.6519   8 307    17.4  3.95 48.3
## 19 22  5.86    0 0.431 6.957   6.8 8.9067   7 330    19.1  3.53 29.6
## 20 90  2.97    0 0.400 7.088  20.8 7.3073   1 285    15.3  7.85 32.2
## 21 80  1.76    0 0.385 6.230  31.5 9.0892   1 241    18.2 12.93 20.1
## 22 33  2.18    0 0.472 6.616  58.1 3.3700   7 222    18.4  8.93 28.4
## 23  0  9.90    0 0.544 6.122  52.8 2.6403   4 304    18.4  5.98 22.1
## 24  0  7.38    0 0.493 6.415  40.1 4.7211   5 287    19.6  6.12 25.0
## 25  0  7.38    0 0.493 6.312  28.9 5.4159   5 287    19.6  6.15 23.0
## 26  0  5.19    0 0.515 5.895  59.6 5.6150   5 224    20.2 10.56 18.5
## 27 80  2.01    0 0.435 6.635  29.7 8.3440   4 280    17.0  5.99 24.5
## 28  0 18.10    0 0.718 3.561  87.9 1.6132  24 666    20.2  7.12 27.5
## 29  0 18.10    1 0.631 7.016  97.5 1.2024  24 666    20.2  2.96 50.0
## 30  0 18.10    0 0.584 6.348  86.1 2.0527  24 666    20.2 17.64 14.5
## 31  0 18.10    0 0.740 5.935  87.9 1.8206  24 666    20.2 34.02  8.4
## 32  0 18.10    0 0.740 5.627  93.9 1.8172  24 666    20.2 22.88 12.8
## 33  0 18.10    0 0.740 5.818  92.4 1.8662  24 666    20.2 22.11 10.5
## 34  0 18.10    0 0.740 6.219 100.0 2.0048  24 666    20.2 16.59 18.4
## 35  0 18.10    0 0.740 5.854  96.6 1.8956  24 666    20.2 23.79 10.8
## 36  0 18.10    0 0.713 6.525  86.5 2.4358  24 666    20.2 18.13 14.1
## 37  0 18.10    0 0.713 6.376  88.4 2.5671  24 666    20.2 14.65 17.7
## 38  0 18.10    0 0.655 6.209  65.4 2.9634  24 666    20.2 13.22 21.4
## 39  0  9.69    0 0.585 5.794  70.6 2.8927   6 391    19.2 14.10 18.3
## 40  0 11.93    0 0.573 6.976  91.0 2.1675   1 273    21.0  5.64 23.9
##      prob predict
## 1  0.0000       0
## 2  0.8258       1
## 3  0.8258       1
## 4  0.8258       1
## 5  0.0690       0
## 6  0.1620       0
## 7  0.1620       0
## 8  0.0056       0
## 9  0.0056       0
## 10 0.0000       0
## 11 0.0000       0
## 12 0.0000       0
## 13 0.9867       1
## 14 0.7985       1
## 15 0.7985       1
## 16 0.3263       0
## 17 0.9558       1
## 18 0.9624       1
## 19 0.0457       0
## 20 0.0000       0
## 21 0.0000       0
## 22 0.5112       1
## 23 0.8747       1
## 24 0.0443       0
## 25 0.0443       0
## 26 0.3074       0
## 27 0.0000       0
## 28 1.0000       1
## 29 1.0000       1
## 30 1.0000       1
## 31 1.0000       1
## 32 1.0000       1
## 33 1.0000       1
## 34 1.0000       1
## 35 1.0000       1
## 36 1.0000       1
## 37 1.0000       1
## 38 1.0000       1
## 39 0.2591       0
## 40 0.0000       0
\end{verbatim}

Split between predicted outcomes is illustrated by tables below.

\begin{verbatim}
## 
##  0  1 
## 19 21
\end{verbatim}

\begin{verbatim}
## 
##     0     1 
## 0.475 0.525
\end{verbatim}

\hypertarget{appendix-b-r-script}{%
\subsection{APPENDIX B: R Script}\label{appendix-b-r-script}}

\begin{Shaded}
\begin{Highlighting}[]
\CommentTok{# Required libraries}
\KeywordTok{library}\NormalTok{(knitr)}
\KeywordTok{library}\NormalTok{(kableExtra)}
\KeywordTok{library}\NormalTok{(ggplot2)}
\KeywordTok{library}\NormalTok{(gridExtra)}
\KeywordTok{library}\NormalTok{(dplyr)}
\KeywordTok{library}\NormalTok{(caTools)}
\KeywordTok{library}\NormalTok{(pscl)}
\KeywordTok{library}\NormalTok{(ROCR)}
\KeywordTok{library}\NormalTok{(MASS)}
\KeywordTok{library}\NormalTok{(caret)}
\KeywordTok{library}\NormalTok{(car)}
\CommentTok{# Import data}
\NormalTok{crime <-}\StringTok{ }\KeywordTok{read.csv}\NormalTok{(}\KeywordTok{url}\NormalTok{(}\KeywordTok{paste0}\NormalTok{(}\StringTok{"https://raw.githubusercontent.com/omerozeren/DATA621/master/crime-training-data_modified.csv"}\NormalTok{)))}
\CommentTok{# Basic statistic}
\KeywordTok{nrow}\NormalTok{(crime); }\KeywordTok{ncol}\NormalTok{(crime)}
\KeywordTok{summary}\NormalTok{(crime)}
\CommentTok{# Summary table}
\NormalTok{sumCrime =}\StringTok{ }\KeywordTok{data.frame}\NormalTok{(}\DataTypeTok{Variable =} \KeywordTok{character}\NormalTok{(),}
                      \DataTypeTok{Min =} \KeywordTok{integer}\NormalTok{(),}
                      \DataTypeTok{Median =} \KeywordTok{integer}\NormalTok{(),}
                      \DataTypeTok{Mean =} \KeywordTok{double}\NormalTok{(),}
                      \DataTypeTok{SD =} \KeywordTok{double}\NormalTok{(),}
                      \DataTypeTok{Max =} \KeywordTok{integer}\NormalTok{(),}
                      \DataTypeTok{Num_NAs =} \KeywordTok{integer}\NormalTok{(),}
                      \DataTypeTok{Num_Zeros =} \KeywordTok{integer}\NormalTok{())}
\ControlFlowTok{for}\NormalTok{ (i }\ControlFlowTok{in} \DecValTok{1}\OperatorTok{:}\DecValTok{13}\NormalTok{) \{}
\NormalTok{  sumCrime <-}\StringTok{ }\KeywordTok{rbind}\NormalTok{(sumCrime, }\KeywordTok{data.frame}\NormalTok{(}\DataTypeTok{Variable =} \KeywordTok{colnames}\NormalTok{(crime)[i],}
                                         \DataTypeTok{Min =} \KeywordTok{min}\NormalTok{(crime[,i], }\DataTypeTok{na.rm=}\OtherTok{TRUE}\NormalTok{),}
                                         \DataTypeTok{Median =} \KeywordTok{median}\NormalTok{(crime[,i], }\DataTypeTok{na.rm=}\OtherTok{TRUE}\NormalTok{),}
                                         \DataTypeTok{Mean =} \KeywordTok{mean}\NormalTok{(crime[,i], }\DataTypeTok{na.rm=}\OtherTok{TRUE}\NormalTok{),}
                                         \DataTypeTok{SD =} \KeywordTok{sd}\NormalTok{(crime[,i], }\DataTypeTok{na.rm=}\OtherTok{TRUE}\NormalTok{),}
                                         \DataTypeTok{Max =} \KeywordTok{max}\NormalTok{(crime[,i], }\DataTypeTok{na.rm=}\OtherTok{TRUE}\NormalTok{),}
                                         \DataTypeTok{Num_NAs =} \KeywordTok{sum}\NormalTok{(}\KeywordTok{is.na}\NormalTok{(crime[,i])),}
                                         \DataTypeTok{Num_Zeros =} \KeywordTok{length}\NormalTok{(}\KeywordTok{which}\NormalTok{(crime[,i]}\OperatorTok{==}\DecValTok{0}\NormalTok{)))}
\NormalTok{  )}
\NormalTok{\}}
\KeywordTok{colnames}\NormalTok{(sumCrime) <-}\StringTok{ }\KeywordTok{c}\NormalTok{(}\StringTok{""}\NormalTok{, }\StringTok{"Min"}\NormalTok{, }\StringTok{"Median"}\NormalTok{, }\StringTok{"Mean"}\NormalTok{, }\StringTok{"SD"}\NormalTok{, }\StringTok{"Max"}\NormalTok{, }
                        \StringTok{"Num of NAs"}\NormalTok{, }\StringTok{"Num of Zeros"}\NormalTok{)}
\NormalTok{sumCrime}
\CommentTok{# Proportion of target variable}
\KeywordTok{table}\NormalTok{(crime}\OperatorTok{$}\NormalTok{target)}
\KeywordTok{table}\NormalTok{(crime}\OperatorTok{$}\NormalTok{target)}\OperatorTok{/}\KeywordTok{sum}\NormalTok{(}\KeywordTok{table}\NormalTok{(crime}\OperatorTok{$}\NormalTok{target))}
\CommentTok{# Exploratory plots (repeated for each variable)}
\KeywordTok{kable}\NormalTok{(sumCrime[sumCrime[,}\DecValTok{1}\NormalTok{]}\OperatorTok{==}\StringTok{"zn"}\NormalTok{,}\DecValTok{2}\OperatorTok{:}\DecValTok{8}\NormalTok{], }\DataTypeTok{row.names=}\OtherTok{FALSE}\NormalTok{)}
\CommentTok{# Get descriptive plots:}
\CommentTok{# Variables: zn indus chas nox rm age dis rad tax ptratio lstat medv target}
\NormalTok{v <-}\StringTok{ "dis"} \CommentTok{# Variable to view}
\NormalTok{pd <-}\StringTok{ }\KeywordTok{as.data.frame}\NormalTok{(}\KeywordTok{cbind}\NormalTok{(crime[, v], crime}\OperatorTok{$}\NormalTok{target))}
\KeywordTok{colnames}\NormalTok{(pd) <-}\StringTok{ }\KeywordTok{c}\NormalTok{(}\StringTok{"X"}\NormalTok{, }\StringTok{"Y"}\NormalTok{)}
\CommentTok{# Boxplot}
\NormalTok{bp <-}\StringTok{ }\KeywordTok{ggplot}\NormalTok{(pd, }\KeywordTok{aes}\NormalTok{(}\DataTypeTok{x =} \DecValTok{1}\NormalTok{, }\DataTypeTok{y =}\NormalTok{ X)) }\OperatorTok{+}\StringTok{ }
\StringTok{  }\KeywordTok{stat_boxplot}\NormalTok{(}\DataTypeTok{geom =}\StringTok{'errorbar'}\NormalTok{) }\OperatorTok{+}\StringTok{ }\KeywordTok{geom_boxplot}\NormalTok{() }\OperatorTok{+}\StringTok{ }
\StringTok{  }\KeywordTok{xlab}\NormalTok{(}\StringTok{"Boxplot"}\NormalTok{) }\OperatorTok{+}\StringTok{ }\KeywordTok{ylab}\NormalTok{(}\StringTok{""}\NormalTok{) }\OperatorTok{+}\StringTok{ }
\StringTok{  }\KeywordTok{theme}\NormalTok{(}\DataTypeTok{axis.text.x=}\KeywordTok{element_blank}\NormalTok{(), }\DataTypeTok{axis.ticks.x=}\KeywordTok{element_blank}\NormalTok{())}
\CommentTok{# Density plot}
\NormalTok{hp <-}\StringTok{ }\KeywordTok{ggplot}\NormalTok{(pd, }\KeywordTok{aes}\NormalTok{(}\DataTypeTok{x =}\NormalTok{ X)) }\OperatorTok{+}\StringTok{ }
\StringTok{  }\KeywordTok{geom_histogram}\NormalTok{(}\KeywordTok{aes}\NormalTok{(}\DataTypeTok{y=}\NormalTok{..density..), }\DataTypeTok{colour=}\StringTok{"black"}\NormalTok{, }\DataTypeTok{fill=}\StringTok{"white"}\NormalTok{) }\OperatorTok{+}
\StringTok{  }\KeywordTok{geom_density}\NormalTok{(}\DataTypeTok{alpha=}\NormalTok{.}\DecValTok{2}\NormalTok{, }\DataTypeTok{fill=}\StringTok{"#FF6666"}\NormalTok{) }\OperatorTok{+}\StringTok{ }
\StringTok{  }\KeywordTok{ylab}\NormalTok{(}\StringTok{""}\NormalTok{) }\OperatorTok{+}\StringTok{ }\KeywordTok{xlab}\NormalTok{(}\StringTok{"Density Plot with Mean"}\NormalTok{) }\OperatorTok{+}
\StringTok{  }\KeywordTok{geom_vline}\NormalTok{(}\KeywordTok{aes}\NormalTok{(}\DataTypeTok{xintercept=}\KeywordTok{mean}\NormalTok{(X, }\DataTypeTok{na.rm=}\OtherTok{TRUE}\NormalTok{)), }
             \DataTypeTok{color=}\StringTok{"red"}\NormalTok{, }\DataTypeTok{linetype=}\StringTok{"dashed"}\NormalTok{, }\DataTypeTok{size=}\DecValTok{1}\NormalTok{)}
\CommentTok{# Scatterplot}
\NormalTok{sp <-}\StringTok{ }\KeywordTok{ggplot}\NormalTok{(pd, }\KeywordTok{aes}\NormalTok{(}\DataTypeTok{x=}\NormalTok{X, }\DataTypeTok{y=}\NormalTok{Y)) }\OperatorTok{+}\StringTok{ }
\StringTok{  }\KeywordTok{geom_point}\NormalTok{() }\OperatorTok{+}\StringTok{   }
\StringTok{  }\KeywordTok{stat_smooth}\NormalTok{(}\DataTypeTok{method=}\StringTok{"glm"}\NormalTok{, }\DataTypeTok{method.args=}\KeywordTok{list}\NormalTok{(}\DataTypeTok{family=}\StringTok{"binomial"}\NormalTok{), }\DataTypeTok{se=}\OtherTok{FALSE}\NormalTok{) }\OperatorTok{+}
\StringTok{  }\KeywordTok{xlab}\NormalTok{(}\StringTok{"Scatterplot with Logistic Regression Line"}\NormalTok{)}
\KeywordTok{grid.arrange}\NormalTok{(bp, hp, sp, }\DataTypeTok{layout_matrix=}\KeywordTok{rbind}\NormalTok{(}\KeywordTok{c}\NormalTok{(}\DecValTok{1}\NormalTok{,}\DecValTok{2}\NormalTok{,}\DecValTok{2}\NormalTok{),}\KeywordTok{c}\NormalTok{(}\DecValTok{1}\NormalTok{,}\DecValTok{3}\NormalTok{,}\DecValTok{3}\NormalTok{)))}
\CommentTok{# Correlation matrix}
\NormalTok{cm <-}\StringTok{ }\KeywordTok{cor}\NormalTok{(crime, }\DataTypeTok{use=}\StringTok{"pairwise.complete.obs"}\NormalTok{)}
\NormalTok{cm <-}\StringTok{ }\KeywordTok{round}\NormalTok{(cm, }\DecValTok{2}\NormalTok{)}
\NormalTok{cmout <-}\StringTok{ }\KeywordTok{as.data.frame}\NormalTok{(cm) }\OperatorTok\StringTok{ }\KeywordTok{mutate_all}\NormalTok{(}\ControlFlowTok{function}\NormalTok{(x) \{}
  \KeywordTok{cell_spec}\NormalTok{(x, }\StringTok{"html"}\NormalTok{, }\DataTypeTok{color =} \KeywordTok{ifelse}\NormalTok{(x}\OperatorTok{>}\FloatTok{0.5} \OperatorTok{|}\StringTok{ }\NormalTok{x}\OperatorTok{<}\NormalTok{(}\OperatorTok{-}\FloatTok{0.5}\NormalTok{),}\StringTok{"blue"}\NormalTok{,}\StringTok{"black"}\NormalTok{))}
\NormalTok{  \})}
\KeywordTok{rownames}\NormalTok{(cmout) <-}\StringTok{ }\KeywordTok{colnames}\NormalTok{(cmout)}
\NormalTok{cmout }\OperatorTok
\StringTok{  }\KeywordTok{kable}\NormalTok{(}\StringTok{"html"}\NormalTok{, }\DataTypeTok{escape =}\NormalTok{ F, }\DataTypeTok{align =} \StringTok{"c"}\NormalTok{, }\DataTypeTok{row.names =} \OtherTok{TRUE}\NormalTok{) }\OperatorTok
\StringTok{  }\KeywordTok{kable_styling}\NormalTok{(}\StringTok{"striped"}\NormalTok{, }\DataTypeTok{full_width =}\NormalTok{ F)}
\KeywordTok{pairs}\NormalTok{(crime)}
\CommentTok{# Force categorical variables to factors}
\NormalTok{crime[,}\StringTok{'rad'}\NormalTok{] <-}\StringTok{ }\KeywordTok{as.factor}\NormalTok{(crime[,}\StringTok{'rad'}\NormalTok{])}
\NormalTok{crime[,}\StringTok{'chas'}\NormalTok{] <-}\StringTok{ }\KeywordTok{as.factor}\NormalTok{(crime[,}\StringTok{'chas'}\NormalTok{])}
\CommentTok{# Split into train and validation sets}
\KeywordTok{set.seed}\NormalTok{(}\DecValTok{88}\NormalTok{)}
\NormalTok{split <-}\StringTok{ }\KeywordTok{sample.split}\NormalTok{(crime}\OperatorTok{$}\NormalTok{target, }\DataTypeTok{SplitRatio =} \FloatTok{0.75}\NormalTok{)}
\NormalTok{crimeTRAIN <-}\StringTok{ }\KeywordTok{subset}\NormalTok{(crime, split }\OperatorTok{==}\StringTok{ }\OtherTok{TRUE}\NormalTok{)}
\NormalTok{crimeTEST <-}\StringTok{ }\KeywordTok{subset}\NormalTok{(crime, split }\OperatorTok{==}\StringTok{ }\OtherTok{FALSE}\NormalTok{)}
\CommentTok{# Model 1}
\NormalTok{model <-}\StringTok{ }\KeywordTok{glm}\NormalTok{ (target }\OperatorTok{~}\StringTok{ }\NormalTok{., }\DataTypeTok{data =}\NormalTok{ crimeTRAIN, }
              \DataTypeTok{family =} \KeywordTok{binomial}\NormalTok{(}\DataTypeTok{link=}\StringTok{"logit"}\NormalTok{))}
\KeywordTok{summary}\NormalTok{(model)}
\NormalTok{pred <-}\StringTok{ }\KeywordTok{predict}\NormalTok{(model, }\DataTypeTok{newdata=}\KeywordTok{subset}\NormalTok{(crimeTEST, }\DataTypeTok{select=}\KeywordTok{c}\NormalTok{(}\DecValTok{1}\OperatorTok{:}\DecValTok{12}\NormalTok{)), }
                \DataTypeTok{type=}\StringTok{'response'}\NormalTok{)}
\NormalTok{cm <-}\StringTok{ }\KeywordTok{confusionMatrix}\NormalTok{(}\KeywordTok{as.factor}\NormalTok{(crimeTEST}\OperatorTok{$}\NormalTok{target), }
                      \KeywordTok{as.factor}\NormalTok{(}\KeywordTok{ifelse}\NormalTok{(pred }\OperatorTok{>}\StringTok{ }\FloatTok{0.5}\NormalTok{,}\DecValTok{1}\NormalTok{,}\DecValTok{0}\NormalTok{)))}
\NormalTok{cm}\OperatorTok{$}\NormalTok{table}
\NormalTok{cm}\OperatorTok{$}\NormalTok{overall[}\StringTok{'Accuracy'}\NormalTok{]}
\KeywordTok{pR2}\NormalTok{(model) }\CommentTok{# McFadden R^2}
\CommentTok{# ROC}
\NormalTok{pr <-}\StringTok{ }\KeywordTok{prediction}\NormalTok{(pred, crimeTEST}\OperatorTok{$}\NormalTok{target)}
\NormalTok{prf <-}\StringTok{ }\KeywordTok{performance}\NormalTok{(pr, }\DataTypeTok{measure =} \StringTok{"tpr"}\NormalTok{, }\DataTypeTok{x.measure =} \StringTok{"fpr"}\NormalTok{)}
\KeywordTok{plot}\NormalTok{(prf, }\DataTypeTok{colorize =} \OtherTok{TRUE}\NormalTok{, }\DataTypeTok{text.adj =} \KeywordTok{c}\NormalTok{(}\OperatorTok{-}\FloatTok{0.2}\NormalTok{,}\FloatTok{1.7}\NormalTok{))}
\NormalTok{auc <-}\StringTok{ }\KeywordTok{performance}\NormalTok{(pr, }\DataTypeTok{measure =} \StringTok{"auc"}\NormalTok{)}
\NormalTok{(auc <-}\StringTok{ }\NormalTok{auc}\OperatorTok{@}\NormalTok{y.values[[}\DecValTok{1}\NormalTok{]])}
\CommentTok{# Model 2}
\NormalTok{model <-}\StringTok{ }\KeywordTok{glm}\NormalTok{ (target }\OperatorTok{~}\StringTok{ }\NormalTok{., }\DataTypeTok{data =}\NormalTok{ crimeTRAIN, }
              \DataTypeTok{family =} \KeywordTok{binomial}\NormalTok{(}\DataTypeTok{link=}\StringTok{"logit"}\NormalTok{))}
\KeywordTok{summary}\NormalTok{(model)}
\NormalTok{model <-}\StringTok{ }\KeywordTok{glm}\NormalTok{ (target }\OperatorTok{~}\StringTok{ }\NormalTok{.}\OperatorTok{-}\NormalTok{rm, }\DataTypeTok{data =}\NormalTok{ crimeTRAIN, }
              \DataTypeTok{family =} \KeywordTok{binomial}\NormalTok{(}\DataTypeTok{link=}\StringTok{"logit"}\NormalTok{))}
\KeywordTok{summary}\NormalTok{(model)}
\NormalTok{model <-}\StringTok{ }\KeywordTok{glm}\NormalTok{ (target }\OperatorTok{~}\StringTok{ }\NormalTok{.}\OperatorTok{-}\NormalTok{rm}\OperatorTok{-}\NormalTok{chas, }\DataTypeTok{data =}\NormalTok{ crimeTRAIN, }
              \DataTypeTok{family =} \KeywordTok{binomial}\NormalTok{(}\DataTypeTok{link=}\StringTok{"logit"}\NormalTok{))}
\KeywordTok{summary}\NormalTok{(model)}
\NormalTok{model <-}\StringTok{ }\KeywordTok{glm}\NormalTok{ (target }\OperatorTok{~}\StringTok{ }\NormalTok{.}\OperatorTok{-}\NormalTok{rm}\OperatorTok{-}\NormalTok{chas}\OperatorTok{-}\NormalTok{lstat, }\DataTypeTok{data =}\NormalTok{ crimeTRAIN, }
              \DataTypeTok{family =} \KeywordTok{binomial}\NormalTok{(}\DataTypeTok{link=}\StringTok{"logit"}\NormalTok{))}
\KeywordTok{summary}\NormalTok{(model)}
\NormalTok{model <-}\StringTok{ }\KeywordTok{glm}\NormalTok{ (target }\OperatorTok{~}\StringTok{ }\NormalTok{.}\OperatorTok{-}\NormalTok{rm}\OperatorTok{-}\NormalTok{chas}\OperatorTok{-}\NormalTok{lstat}\OperatorTok{-}\NormalTok{indus, }\DataTypeTok{data =}\NormalTok{ crimeTRAIN, }
              \DataTypeTok{family =} \KeywordTok{binomial}\NormalTok{(}\DataTypeTok{link=}\StringTok{"logit"}\NormalTok{))}
\KeywordTok{summary}\NormalTok{(model)}
\NormalTok{model <-}\StringTok{ }\KeywordTok{glm}\NormalTok{ (target }\OperatorTok{~}\StringTok{ }\NormalTok{.}\OperatorTok{-}\NormalTok{rm}\OperatorTok{-}\NormalTok{chas}\OperatorTok{-}\NormalTok{lstat}\OperatorTok{-}\NormalTok{indus}\OperatorTok{-}\NormalTok{zn, }\DataTypeTok{data =}\NormalTok{ crimeTRAIN, }
              \DataTypeTok{family =} \KeywordTok{binomial}\NormalTok{(}\DataTypeTok{link=}\StringTok{"logit"}\NormalTok{))}
\KeywordTok{summary}\NormalTok{(model)}
\NormalTok{pred <-}\StringTok{ }\KeywordTok{predict}\NormalTok{(model, }\DataTypeTok{newdata=}\KeywordTok{subset}\NormalTok{(crimeTEST, }\DataTypeTok{select=}\KeywordTok{c}\NormalTok{(}\DecValTok{1}\OperatorTok{:}\DecValTok{12}\NormalTok{)), }
                \DataTypeTok{type=}\StringTok{'response'}\NormalTok{)}
\NormalTok{cm <-}\StringTok{ }\KeywordTok{confusionMatrix}\NormalTok{(}\KeywordTok{as.factor}\NormalTok{(crimeTEST}\OperatorTok{$}\NormalTok{target), }
                      \KeywordTok{as.factor}\NormalTok{(}\KeywordTok{ifelse}\NormalTok{(pred }\OperatorTok{>}\StringTok{ }\FloatTok{0.5}\NormalTok{,}\DecValTok{1}\NormalTok{,}\DecValTok{0}\NormalTok{)))}
\NormalTok{cm}\OperatorTok{$}\NormalTok{table}
\NormalTok{cm}\OperatorTok{$}\NormalTok{overall[}\StringTok{'Accuracy'}\NormalTok{]}
\KeywordTok{pR2}\NormalTok{(model) }\CommentTok{# McFadden R^2}
\CommentTok{# ROC}
\NormalTok{pr <-}\StringTok{ }\KeywordTok{prediction}\NormalTok{(pred, crimeTEST}\OperatorTok{$}\NormalTok{target)}
\NormalTok{prf <-}\StringTok{ }\KeywordTok{performance}\NormalTok{(pr, }\DataTypeTok{measure =} \StringTok{"tpr"}\NormalTok{, }\DataTypeTok{x.measure =} \StringTok{"fpr"}\NormalTok{)}
\KeywordTok{plot}\NormalTok{(prf, }\DataTypeTok{colorize =} \OtherTok{TRUE}\NormalTok{, }\DataTypeTok{text.adj =} \KeywordTok{c}\NormalTok{(}\OperatorTok{-}\FloatTok{0.2}\NormalTok{,}\FloatTok{1.7}\NormalTok{))}
\NormalTok{auc <-}\StringTok{ }\KeywordTok{performance}\NormalTok{(pr, }\DataTypeTok{measure =} \StringTok{"auc"}\NormalTok{)}
\NormalTok{(auc <-}\StringTok{ }\NormalTok{auc}\OperatorTok{@}\NormalTok{y.values[[}\DecValTok{1}\NormalTok{]])}
\CommentTok{# Take out 'tax' because it is highly correlated with 'rad'}
\NormalTok{model <-}\StringTok{ }\KeywordTok{glm}\NormalTok{ (target }\OperatorTok{~}\StringTok{ }\NormalTok{.}\OperatorTok{-}\NormalTok{tax, }\DataTypeTok{data =}\NormalTok{ crimeTRAIN, }
              \DataTypeTok{family =} \KeywordTok{binomial}\NormalTok{(}\DataTypeTok{link=}\StringTok{"logit"}\NormalTok{))}
\KeywordTok{summary}\NormalTok{(model)}
\NormalTok{pred <-}\StringTok{ }\KeywordTok{predict}\NormalTok{(model, }\DataTypeTok{newdata=}\KeywordTok{subset}\NormalTok{(crimeTEST, }\DataTypeTok{select=}\KeywordTok{c}\NormalTok{(}\DecValTok{1}\OperatorTok{:}\DecValTok{12}\NormalTok{)), }
                \DataTypeTok{type=}\StringTok{'response'}\NormalTok{)}
\NormalTok{cm <-}\StringTok{ }\KeywordTok{confusionMatrix}\NormalTok{(}\KeywordTok{as.factor}\NormalTok{(crimeTEST}\OperatorTok{$}\NormalTok{target), }
                      \KeywordTok{as.factor}\NormalTok{(}\KeywordTok{ifelse}\NormalTok{(pred }\OperatorTok{>}\StringTok{ }\FloatTok{0.5}\NormalTok{,}\DecValTok{1}\NormalTok{,}\DecValTok{0}\NormalTok{)))}
\NormalTok{cm}\OperatorTok{$}\NormalTok{table}
\NormalTok{cm}\OperatorTok{$}\NormalTok{overall[}\StringTok{'Accuracy'}\NormalTok{]}
\KeywordTok{pR2}\NormalTok{(model) }\CommentTok{# McFadden R^2}
\CommentTok{# ROC}
\NormalTok{pr <-}\StringTok{ }\KeywordTok{prediction}\NormalTok{(pred, crimeTEST}\OperatorTok{$}\NormalTok{target)}
\NormalTok{prf <-}\StringTok{ }\KeywordTok{performance}\NormalTok{(pr, }\DataTypeTok{measure =} \StringTok{"tpr"}\NormalTok{, }\DataTypeTok{x.measure =} \StringTok{"fpr"}\NormalTok{)}
\KeywordTok{plot}\NormalTok{(prf, }\DataTypeTok{colorize =} \OtherTok{TRUE}\NormalTok{, }\DataTypeTok{text.adj =} \KeywordTok{c}\NormalTok{(}\OperatorTok{-}\FloatTok{0.2}\NormalTok{,}\FloatTok{1.7}\NormalTok{))}
\NormalTok{auc <-}\StringTok{ }\KeywordTok{performance}\NormalTok{(pr, }\DataTypeTok{measure =} \StringTok{"auc"}\NormalTok{)}
\NormalTok{(auc <-}\StringTok{ }\NormalTok{auc}\OperatorTok{@}\NormalTok{y.values[[}\DecValTok{1}\NormalTok{]])}
\CommentTok{# Slight improvement}
\CommentTok{# Step AIC method}
\NormalTok{model <-}\StringTok{ }\KeywordTok{glm}\NormalTok{ (target }\OperatorTok{~}\StringTok{ }\NormalTok{., }\DataTypeTok{data =}\NormalTok{ crimeTRAIN, }
              \DataTypeTok{family =} \KeywordTok{binomial}\NormalTok{(}\DataTypeTok{link=}\StringTok{"logit"}\NormalTok{))}
\NormalTok{model <-}\StringTok{ }\KeywordTok{stepAIC}\NormalTok{(model, }\DataTypeTok{trace=}\OtherTok{TRUE}\NormalTok{)}
\KeywordTok{summary}\NormalTok{(model)}
\NormalTok{pred <-}\StringTok{ }\KeywordTok{predict}\NormalTok{(model, }\DataTypeTok{newdata=}\KeywordTok{subset}\NormalTok{(crimeTEST, }\DataTypeTok{select=}\KeywordTok{c}\NormalTok{(}\DecValTok{1}\OperatorTok{:}\DecValTok{12}\NormalTok{)), }
                \DataTypeTok{type=}\StringTok{'response'}\NormalTok{)}
\NormalTok{cm <-}\StringTok{ }\KeywordTok{confusionMatrix}\NormalTok{(}\KeywordTok{as.factor}\NormalTok{(crimeTEST}\OperatorTok{$}\NormalTok{target), }
                      \KeywordTok{as.factor}\NormalTok{(}\KeywordTok{ifelse}\NormalTok{(pred }\OperatorTok{>}\StringTok{ }\FloatTok{0.5}\NormalTok{,}\DecValTok{1}\NormalTok{,}\DecValTok{0}\NormalTok{)))}
\NormalTok{cm}\OperatorTok{$}\NormalTok{table}
\NormalTok{cm}\OperatorTok{$}\NormalTok{overall[}\StringTok{'Accuracy'}\NormalTok{]}
\KeywordTok{pR2}\NormalTok{(model) }\CommentTok{# McFadden R^2}
\CommentTok{# ROC}
\NormalTok{pr <-}\StringTok{ }\KeywordTok{prediction}\NormalTok{(pred, crimeTEST}\OperatorTok{$}\NormalTok{target)}
\NormalTok{prf <-}\StringTok{ }\KeywordTok{performance}\NormalTok{(pr, }\DataTypeTok{measure =} \StringTok{"tpr"}\NormalTok{, }\DataTypeTok{x.measure =} \StringTok{"fpr"}\NormalTok{)}
\KeywordTok{plot}\NormalTok{(prf, }\DataTypeTok{colorize =} \OtherTok{TRUE}\NormalTok{, }\DataTypeTok{text.adj =} \KeywordTok{c}\NormalTok{(}\OperatorTok{-}\FloatTok{0.2}\NormalTok{,}\FloatTok{1.7}\NormalTok{))}
\NormalTok{auc <-}\StringTok{ }\KeywordTok{performance}\NormalTok{(pr, }\DataTypeTok{measure =} \StringTok{"auc"}\NormalTok{)}
\NormalTok{(auc <-}\StringTok{ }\NormalTok{auc}\OperatorTok{@}\NormalTok{y.values[[}\DecValTok{1}\NormalTok{]])}
\CommentTok{# Bad model for testing of code}
\NormalTok{model <-}\StringTok{ }\KeywordTok{glm}\NormalTok{ (target }\OperatorTok{~}\StringTok{ }\NormalTok{age}\OperatorTok{+}\NormalTok{tax, }\DataTypeTok{data =}\NormalTok{ crimeTRAIN, }
              \DataTypeTok{family =} \KeywordTok{binomial}\NormalTok{(}\DataTypeTok{link=}\StringTok{"logit"}\NormalTok{))}
\KeywordTok{summary}\NormalTok{(model)}
\NormalTok{pred <-}\StringTok{ }\KeywordTok{predict}\NormalTok{(model, }\DataTypeTok{newdata=}\KeywordTok{subset}\NormalTok{(crimeTEST, }\DataTypeTok{select=}\KeywordTok{c}\NormalTok{(}\DecValTok{1}\OperatorTok{:}\DecValTok{12}\NormalTok{)), }
                \DataTypeTok{type=}\StringTok{'response'}\NormalTok{)}
\NormalTok{cm <-}\StringTok{ }\KeywordTok{confusionMatrix}\NormalTok{(}\KeywordTok{as.factor}\NormalTok{(crimeTEST}\OperatorTok{$}\NormalTok{target), }
                      \KeywordTok{as.factor}\NormalTok{(}\KeywordTok{ifelse}\NormalTok{(pred }\OperatorTok{>}\StringTok{ }\FloatTok{0.5}\NormalTok{,}\DecValTok{1}\NormalTok{,}\DecValTok{0}\NormalTok{)))}
\NormalTok{cm}\OperatorTok{$}\NormalTok{table}
\NormalTok{cm}\OperatorTok{$}\NormalTok{overall[}\StringTok{'Accuracy'}\NormalTok{]}
\KeywordTok{pR2}\NormalTok{(model) }\CommentTok{# McFadden R^2}
\CommentTok{# ROC}
\NormalTok{pr <-}\StringTok{ }\KeywordTok{prediction}\NormalTok{(pred, crimeTEST}\OperatorTok{$}\NormalTok{target)}
\NormalTok{prf <-}\StringTok{ }\KeywordTok{performance}\NormalTok{(pr, }\DataTypeTok{measure =} \StringTok{"tpr"}\NormalTok{, }\DataTypeTok{x.measure =} \StringTok{"fpr"}\NormalTok{)}
\KeywordTok{plot}\NormalTok{(prf, }\DataTypeTok{colorize =} \OtherTok{TRUE}\NormalTok{, }\DataTypeTok{text.adj =} \KeywordTok{c}\NormalTok{(}\OperatorTok{-}\FloatTok{0.2}\NormalTok{,}\FloatTok{1.7}\NormalTok{))}
\NormalTok{auc <-}\StringTok{ }\KeywordTok{performance}\NormalTok{(pr, }\DataTypeTok{measure =} \StringTok{"auc"}\NormalTok{)}
\NormalTok{(auc <-}\StringTok{ }\NormalTok{auc}\OperatorTok{@}\NormalTok{y.values[[}\DecValTok{1}\NormalTok{]])}
\CommentTok{# Selected model}
\NormalTok{model <-}\StringTok{ }\KeywordTok{glm}\NormalTok{(target }\OperatorTok{~}\StringTok{ }\NormalTok{zn}\OperatorTok{+}\KeywordTok{I}\NormalTok{(nox}\OperatorTok{*}\DecValTok{100}\NormalTok{)}\OperatorTok{+}\NormalTok{rad}\OperatorTok{+}\NormalTok{tax}\OperatorTok{+}\NormalTok{indus, }\DataTypeTok{data =}\NormalTok{ crimeTRAIN, }
             \DataTypeTok{family =} \KeywordTok{binomial}\NormalTok{(}\DataTypeTok{link =} \StringTok{"logit"}\NormalTok{))}
\CommentTok{# K-Fold cross validation}
\NormalTok{ctrl <-}\StringTok{ }\KeywordTok{trainControl}\NormalTok{(}\DataTypeTok{method =} \StringTok{"repeatedcv"}\NormalTok{, }\DataTypeTok{number =} \DecValTok{10}\NormalTok{, }
                     \DataTypeTok{savePredictions =} \OtherTok{TRUE}\NormalTok{)}
\NormalTok{model_fit <-}\StringTok{ }\KeywordTok{train}\NormalTok{(target }\OperatorTok{~}\StringTok{ }\NormalTok{zn }\OperatorTok{+}\StringTok{ }\NormalTok{nox }\OperatorTok{+}\StringTok{ }\NormalTok{rad }\OperatorTok{+}\StringTok{ }\NormalTok{tax,  }
                   \DataTypeTok{data=}\NormalTok{crimeTRAIN, }\DataTypeTok{method=}\StringTok{"glm"}\NormalTok{, }
                   \DataTypeTok{family=}\StringTok{"binomial"}\NormalTok{,}
                   \DataTypeTok{trControl =}\NormalTok{ ctrl, }\DataTypeTok{tuneLength =} \DecValTok{5}\NormalTok{)}
\NormalTok{pred <-}\StringTok{ }\KeywordTok{predict}\NormalTok{(model_fit, }\DataTypeTok{newdata=}\NormalTok{crimeTEST)}
\KeywordTok{confusionMatrix}\NormalTok{(}\KeywordTok{as.factor}\NormalTok{(crimeTEST}\OperatorTok{$}\NormalTok{target), }
                \KeywordTok{as.factor}\NormalTok{(}\KeywordTok{ifelse}\NormalTok{(pred }\OperatorTok{>}\StringTok{ }\FloatTok{0.5}\NormalTok{,}\DecValTok{1}\NormalTok{,}\DecValTok{0}\NormalTok{)))}
\CommentTok{# Deviance residuals}
\KeywordTok{anova}\NormalTok{(model, }\DataTypeTok{test=}\StringTok{"Chisq"}\NormalTok{)}
\CommentTok{# VIF}
\KeywordTok{vif}\NormalTok{(model)}
\CommentTok{# Prediction}
\NormalTok{eval <-}\StringTok{ }\KeywordTok{read.csv}\NormalTok{(}\StringTok{"https://raw.githubusercontent.com/omerozeren/DATA621/master/crime-evaluation-data_modified.csv"}\NormalTok{)}
\NormalTok{eval[,}\StringTok{'rad'}\NormalTok{] <-}\StringTok{ }\KeywordTok{as.factor}\NormalTok{(eval[,}\StringTok{'rad'}\NormalTok{])}
\NormalTok{eval[,}\StringTok{'chas'}\NormalTok{] <-}\StringTok{ }\KeywordTok{as.factor}\NormalTok{(eval[,}\StringTok{'chas'}\NormalTok{])}
\NormalTok{pred <-}\StringTok{ }\KeywordTok{predict}\NormalTok{(model, }\DataTypeTok{newdata=}\NormalTok{eval, }\DataTypeTok{type=}\StringTok{"response"}\NormalTok{)}
\NormalTok{eval <-}\StringTok{ }\KeywordTok{cbind}\NormalTok{(eval, }\DataTypeTok{prob=}\KeywordTok{round}\NormalTok{(pred,}\DecValTok{4}\NormalTok{))}
\NormalTok{eval <-}\StringTok{ }\KeywordTok{cbind}\NormalTok{(eval, }\DataTypeTok{predict=}\KeywordTok{round}\NormalTok{(pred,}\DecValTok{0}\NormalTok{))}
\KeywordTok{kable}\NormalTok{(eval, }\DataTypeTok{row.names=}\OtherTok{FALSE}\NormalTok{)}
\end{Highlighting}
\end{Shaded}


\end{document}
